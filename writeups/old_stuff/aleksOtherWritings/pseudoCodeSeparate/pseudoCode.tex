

%%% line numbers
%%% make a version that is less formal... and compare which one is better

\documentclass[11pt]{article}
\author{Alek Westover}
\title{Cache Efficient Parallel Partition}

\renewcommand{\thesubsection}{\thesection.\alph{subsection}}
\usepackage{graphicx}
\usepackage{listings}
\usepackage{amsfonts}

\usepackage{amsmath}
% defines a new function
\def\polylog{\operatorname{polylog}}
\def\pred{\operatorname{pred}}
\def\E{\operatorname{\mathbb{E}}}

\usepackage{mathtools}
\DeclarePairedDelimiter{\floor}{\lfloor}{\rfloor}
% \floor*{\frac{x}{y}} actually does a nice looking floor!
\DeclarePairedDelimiter{\paren}{(}{)}
% paren*{ } will also give the correct size
\newcommand{\defn}[1]       {{\textit{\textbf{\boldmath #1}}}}

\renewcommand{\paragraph}[1]{\vspace{0.09in}\noindent{\bf \boldmath #1.}} 

\usepackage{amsthm}
\usepackage{amssymb}
\usepackage{algorithm, algpseudocode}

%%%% Theorem stuff
\newtheorem{theorem}{Theorem}
\newtheorem{lemma}{Lemma}
\newtheorem{proposition}{Proposition}

\begin{document}

\begin{figure*}
	\caption{Parallel Partition}
	\label{alg:parallelPartition}
	\begin{algorithmic}
		\If{$g<2$}
			\State serialPartition A
		\Else
			\For{$i \in \{0,1,\ldots,s-1\}$}
				\State $X[i] \gets$ a random integer from $[0,g-1]$ 
			\EndFor
			\State -- We implement this parallel for-loop with a sequence of recursive spawns, and which facilitates computing $v_{min}$, $v_{max}$ without storing $v_y$s
			\ForAll{$ y \in \{0, 1,\ldots,g-1\}$ in parallel}
				\State -- Now we perform a serial partition on $U_y$

				\State -- Initialize ALowIdx to be the index of the first element in $U_y$
				\State $\text{ALowIdx} \gets ((X[0]+y) \bmod g)\cdot b$

				\State -- Initialize AHighIdx to be the index of the last element in $U_y$
				\State $\text{AHighIdx} \gets n - g\cdot b + ((X[s-1]+y) \bmod g)\cdot b + b-1$

				\While{ALowIdx $<$ AHighIdx} 
					\While{A[ALowIdx] $\leq$ pivotValue}
						\State ALowIdx $\gets$ ALowIdx$+1$
						\If{ALowIdx on block boundary}
							\State -- We perform a block increment
							\State $i \gets $ \# of block increments so far (including this one)
							\State -- Increase ALowIdx to start of block $i$ of $G_y$
							\State $\text{ALowIdx} \gets ((X[i] + y)\bmod g) \cdot b + i\cdot b\cdot g$
						\EndIf
					\EndWhile
					\While{A[high] $>$ pivotValue}
						\State AHighIdx $\gets$ AHighIdx$-1$
						\If{AHighIdx on block boundary}
							\State -- We perform a block decrement
							\State $i \gets $ \# of block decrements so far (including this one)
							\State -- Decrease AHighIdx to end of block $s-1-i$ of $G_y$
							\State $\text{AHighIdx} \gets ((X[s-1-i] + y)\bmod g) \cdot b + i\cdot b\cdot g + b - 1$
						\EndIf
					\EndWhile

					\State Swap $A[\text{ALowIdx}]$ and $A[\text{AHighIdx}]$
				\EndWhile
			\EndFor
			\State Recurse on $A[v_{min}],\ldots,A[v_{max}-1]$
		\EndIf
	\end{algorithmic}	
\end{figure*}

\end{document}


