
% Alek Westover writeup draft 

\documentclass[twocolumn, twoside, 12pt]{article}
\author{\vspace{-2ex}Alek Westover\vspace{-2ex}}
\title{\vspace{-8ex}Cache Efficient Parallel Partition\vspace{-2ex}}
\date{\vspace{-2ex}June 26, 2019\vspace{-2ex}}

\renewcommand{\thesubsection}{\thesection.\alph{subsection}}
\usepackage{graphicx}
\usepackage{listings}
\usepackage{amsfonts}

\usepackage{amsmath}
% defines a new function
\def\polylog{\operatorname{polylog}}
\def\pred{\operatorname{pred}}
\def\E{\operatorname{\mathbb{E}}}
\def\Pr{\operatorname{Pr}}

\usepackage{mathtools}
\DeclarePairedDelimiter{\floor}{\lfloor}{\rfloor}
% \floor*{\frac{x}{y}} actually does a nice looking floor!
\DeclarePairedDelimiter{\paren}{(}{)}
% paren*{ } gives the correct size
\newcommand{\defn}[1]       {{\textit{\textbf{\boldmath #1}}}}

\renewcommand{\paragraph}[1]{\vspace{0.09in}\noindent{\bf \boldmath #1.}} 

\usepackage{amsthm}
\usepackage{amssymb}
\usepackage{algorithm, algpseudocode}
% \usepackage[noend]{algpseudocode} -- this would get rid of the endifs, maybe worth thinking about for code concision

%%%% Theorem stuff
\newtheorem{theorem}{Theorem}
\newtheorem{lemma}{Lemma}
\newtheorem{proposition}{Proposition}


\begin{document}
\maketitle
\begin{abstract}
Quicksort is a very important problem to which a fast solution is desirable. 
By utilizing parallel processing it is possible to perform quicksort much faster than a serial implementation could. 
Quicksort consists of partitioning an array according to some pivot value such that the elements with value at most the pivot value occur before the elements with value greater than the pivot value, and then recursively sorting each of these parts of the array. 
Thus, parallel partition is the important algorithm to make fast in order to make quicksort fast. 
Parallel partition is also an important problem in its own right, which further motivates exploration of parallel algorithms for performing a partition. 
	In this paper we present an algorithm for parallel partition that has low span, is fast in practice, incurs only $n(1+o(1))$ cache misses, is in place, and has theoretical guarantees for arbitrary inputs.
\end{abstract}

\section{Introduction}
\paragraph{Serial Partition Problem}
We describe the serial partition problem and the solution which is used extensively in parallel partition algorithms. 
The serial partition problem takes in an array, and then reorders the array according to a decider function.
The serial partition algorithm, which runs fully in-place, has the end result of the array being partitioned such that all elements labelled predecessors by the decider function occur before the elements labelled successors by the decider function.
In the case of the decider function "$A[i] \leq pivotValue$", as is the case in the partition used for quicksort to sort an array in ascending order, this corresponds to placing all elements with values less than or equal to the pivot value before the elements with values greater than or equal to the pivot value.
Thus, upon recursively sorting the predecessors and the successors the entire sorting problem is solved. This is because the predecessor portion of the array is sorted, the successor portion of the array is sorted, and the predecessors occur before the successors, so the whole array is sorted.
The serial partition algorithm works as follows: 

% \begin{samepage}
\begin{enumerate}
	\item{Initialize \defn{low} to point at the beginning of the array, and initialize \defn{high} to point at the end of the array}
	\item{Increment low until $A[\text{low}]$ is a successor}
	\item{Decrement high until $A[\text{high}]$ is a predecessor}
	\item{Swap values $A[\text{low}]$ and $A[\text{high}]$ in the array}
	\item{Repeat steps 3-5 until $\text{high} \geq \text{low}$ which means that all elements in the array have been processed}
	\item{If $A[\text{low}]$ is a predecessor increment $A[\text{low}]$ by $1$ so that $A[\text{low}]$ is the first successor in $A$, which is now partitioned}
\end{enumerate}
% \end{samepage}

See Algorithm \ref{alg:serialPartition} for a pseudocode implementation of this algorithm. Note that the serial partition described makes a single pass over the $n$ elements in the array meaning that it incurs $O(n)$ cache misses, and also it has running time $O(n)$.

\vspace{5mm}
\begin{samepage}
\begin{algorithm}
	\caption{Serial Partition}
	\label{alg:serialPartition}
	\begin{algorithmic}
		\State $\text{low} \gets 0; \text{high} \gets n-1$
		\While{$\text{low} < \text{high}$} 
			\While{$A[\text{low}] \leq \text{pivotValue}$}
				\State $\text{low} \gets \text{low}+1$
			\EndWhile
			\While{$A[\text{high}] > \text{pivotValue}$}
				\State $high \gets high-1$
			\EndWhile
			\State Swap $A[\text{low}]$ and $A[\text{high}]$
		\EndWhile
		\If{$A[\text{low}] \leq \text{pivotValue}$}
			\State $low \gets low+1$
		\EndIf
	\end{algorithmic}	
\end{algorithm}
\end{samepage}

\paragraph{Further Preliminaries}
Here we introduce some key ideas in analysis of the algorithms presented in this article.
Many of the algorithms we present have worst case performances that are highly unlikely to be realized.
For instance, if in quicksort a random selection of a pivot value results in only 1 element being labelled a successor, then the recursive subproblems are very unevenly sized.
If the size of the problems is $n, n-1, n-2,\ldots,2$ then quicksort performs at its worst case performance, which is $O(n^2)$.
This however is very unlikely.
We can formalize this with the notion of saying that the event does not happen \defn{with high probability}.
An event is said to happen with high probability in $n$ if the probability of the event occurring is $1-n^{-c}$ for some $c>0$.
Intuitively, if we can show that our algorithm has a certain span with high probability, then we are confident that the algorithm achieves this span.
We define throughout our analysis $\mu$ to be the fraction of predecessors in the array $A$.
In implementing the partition algorithm we do not have control over the pivot value, it is an input to our problem.
The pivot value in conjunction with the array $A$ and the decider function determine the fraction $\mu$.
Throughout our analysis we assume that $\mu$ is a constant not dependent on $n$ which would be quite adversarial.

\paragraph{Factors to Consider}
We now describe the desirable characteristics of a parallel partition algorithm.
The \defn{work} of an algorithm, $T_1$, is the running time of the algorithm in serial. 
The \defn{span} of an algorithm, $T_\infty$ is the running time of the algorithm when run on an infinite number of processors.
The running time of the algorithm on $p$ processors, $T_p$, is determined by its work and span.
Brent's theorem states that $$T_p = \Theta \paren*{T_\infty + \frac{T_1}{p}}$$ which shows why optimizing the algorithms span and work will optimize its performance on $p$ processors.
Another important consideration is auxiliary memory usage. It is desirable to be as close to in-place as possible because a user might not have any extra space that our algorithm can use if the array to be partitioned takes up nearly all of their machine's memory.
Finally, it is important to analyze whether or not the algorithm is cache efficient, because this determines the \defn{memory bandwidth bound} on the algorithm.
A \defn{cache miss} is when the algorithm must access a value that is not in the cache already.
Doing this takes time, and the memory banks on a computer can only service a limited number of memory access attempts per interval of time, so if there are lots of requests to the memory bank happening then the algorithm slows down, and we say it is memory bandwidth bound.
For the partition problem each element must be read at least once, so there are at least $1\cdot n$ cache misses.
We consider all of these properties in our development of a parallel partition algorithm, and our algorithm does well according to all of these metrics.

\paragraph{Prior Work}
Prior work has been done on the parallel partition problem.
William Kuszmaul developed an in-place parallel partition algorithm.
Unfortunately, despite theoretical guarantees, his algorithm was outperformed by an optimized but higher span algorithm.
The problem with this parallel partition algorithm in terms of actual speed is memory bandwidth bound.
The reason for the memory bandwidth bound is the number of cache misses that the algorithm incurs.
Thus reducing the number of cache misses is a promising place to look in order to speed up parallel partition, which motivates our algorithm.

	
\section{Algorithm With Desirable Guarantees On Random Inputs}
There is an algorithm for performing parallel partition developed by Francis and Pannan, which we call the \defn{Strided Algorithm}.
The Strided Algorithm is thought to use very few passes over the array, although we could not find any accurate theoretical analysis of the algorithm.
Having a low number of cache misses is desirable because it makes the algorithm run much faster, overcoming the problem of memory bandwidth bound.
The Strided Algorithm does not achieve a small number of cache misses for all inputs, but it does achieve this when used on random inputs.
We now present a proof of the following theorem regarding the algorithm.

\begin{theorem}
Given an array $A[0], A[1], \ldots, A[n-1]$ where each $A[i]$ is chosen randomly independently to be either $0$ or $1$, and a decider function that labels each element with value $0$ a predecessor, and each element with the value $1$ a successor, the Strided Algorithm partitions the array in span that is, with high probability in n, $O(\frac{n}{t}+\sqrt{n\cdot t\cdot\log n})$ where $t$ is a parameter of our choice indicating the number of parts that we break $A$ into, incurring $n\cdot(1+o(1))$ cache misses.
\label{thm:stridedAlg}
\end{theorem}

\paragraph{Definitions}
Now we describe the algorithm.
The algorithm is given an input array $A[0],A[1],\ldots,A[n-1]$ where for all $i$, $A[i] \in \{0, 1\}$.
The values in the array $A$ were determined independently, and randomly with an equal probability of being either $0$ or $1$.
The Strided Algorithm first logically partitions $A$ into parts $P_0, P_1, \ldots, P_{t-1}$ where 
\begin{equation*}
	P_i = \{A[i], A[i+t], \ldots, A[i+(\frac{n}{t}-1)t]\}.
\end{equation*}
Note that we do not need to expend any extra memory to store each part $P_i$ because the part is determined by a formula that can be used to access elements in $P_i$.
The length of $P_i \approx n/t$. This is approximate because if $n \mod t \not\equiv 0$ then some of the $P_i$s have $1$ less element than others.\\
Define $x_i$ to be the number of predecessors in $P_i$.
The algorithm performs a serial partition of each $P_i$ in parallel (on all the $P_is$ at once).
Note that this serial partition is slightly different than a typical serial partition because the elements in $P_i$ are not adjacent in the array $A$, so to go from one value in $P_i$ to the next we must add $t$ rather than simply $1$, and in decrementing we decrement by $t$ instead of by $1$.
Partitioning $P_i$ makes the first $x_i$ elements of $P_i$ predecessors.
Define $v_i = t\cdot x_i+i \approx t\cdot x_i$.
The index $v_i$ is the location in $A$ of the first successor in group $P_i$, because by construction $P_i$ consists of elements that are $t$ apart in the array, so by scaling the index relative to the part $P_i$ by $t$ (and shifting by $i$ so that the first element is aligned) we get the index into $A$.
Define $v_{min} = \min_i{v_i}$ and $v_{max} = \max_i {v_i}$.
Note that after the partitioning all elements in the array $A$ with index less than $v_{min}$ are predecessors, and all elements after and including $v_{max}$ are successors, so to finish the partitioning of the array $A$ we only need to partition a subarray of size $v_{max}-v_{min}$.
Note that we cannot recursively apply the Strided Algorithm to solve the sub problem, because it would behave very poorly on the subarray, and also because, as we demonstrate, the size of the subarray should be very small relative to $n$.
Also note that if we can show that $v_{max} - v_{min}$ is small relative to $n$ then the partition algorithm uses very few passes over the array.
We proceed to prove the following lemma.

\begin{lemma}
With high probability in $n$, the size of the recursive subproblem, $v_{max} - v_{min}$, is $O(\sqrt{t\cdot n\log n}).$
\label{lem:epsSmallFP}
\end{lemma}

\begin{proof}[Proof of Lemma \ref{lem:epsSmallFP}]
$ $\newline
\paragraph{A Single $P_i$}
To prove this, we use Chernoff Bounds, which give bounds on the probability of a deviation of a random variable from its mean (expected value) $\mu$ by more than a certain amount.
We first bound the number of predecessors $x_i$ in each part $P_i$, which allows us to bound $v_i$, the index in the array $A$ of the first successor in $P_i$ after $P_i$ has been partitioned.

The Multiplicative Chernoff bound bounds the probability of a random variable such as $x_i$ deviating by more than $\delta \cdot \mu$ from its mean.
Specifically it says,
$$P(x_i>\mu(1+\delta)) = O(e^{-\mu \cdot \delta^2/3}).$$
We want to know what value of $\delta$ makes $x_i\leq\mu(1+\delta)$ with high probability.
To find the smallest $\delta$ satisfying this, we must solve 
$$O(e^{-\mu \cdot \delta^2/3}) = n^{-c},$$
where the right hand side comes from the definition of an event being false with high probability, and the left hand side comes from the probability guaranteed by the Chernoff bound.
Solving, we obtain 
$$\log e^{-\mu \cdot \delta^2/3} = \log n^{-c},$$
which upon isolating $\delta$ yields,
$$\delta = O\paren*{\sqrt{\frac{\log n}{\mu}}}.$$
Because $x_i$ is the number of predecessors in $P_i$, and $|P_i| \approx \frac{n}{t}$, the expectation of $x_i$ is $\mu = \frac{n}{2t}$ because the values in A were generated independently at random, with $\frac{1}{2}$ probability of being predecessors.
Thus we can assert that 
$$\delta = O\paren*{\sqrt{\frac{t\log n}{n}}}.$$
By a nearly identical argument with the Chernoff bound for the probability of $x_i$ deviating below the mean $\mu$ by more than $\mu \cdot \delta$ we get the same result: that deviation by more than $\mu \cdot \delta$ does not happen with high probability.

Thus we have that with high probability
$$|x_i - \mu| = O\paren*{\sqrt{\frac{t\log n}{n}}} \cdot \mu.$$
Once again we can substitute in $\mu=\frac{n}{2t}$ to get,
$$|x_i - \mu| = O\paren*{\sqrt{\frac{t\log n}{n}}} \cdot \frac{n}{2t}.$$
This simplifies to,
$$\Big|x_i - \frac{n}{2t}\Big| = O\paren*{\sqrt{\frac{n\log n}{t}}}.$$
To convert this index to an index into the array A rather than an index in $P_i$, we multiply by $t$ (the gap between the indices into the array A of elements in $P_i$) obtaining,
$$\Big|t\cdot x_i - \frac{n}{2}\Big| = O(\sqrt{t\cdot n\log n}).$$
Equivalently, we can use our name $v_i$ for the index into the array $A$ to write this as
$$\Big|v_i - \frac{n}{2}\Big| = O(\sqrt{t\cdot n\log n}).$$
We now need to move the discussion from just looking at $P_i$ to looking at all $P_i$s.

\paragraph{Union Bound over all $P_i$}
Now, we apply the union bound for with high probability events.
It states that the union of a set of events that are false with high probability is still false with high probability.
Using the union bound allows us to go from saying 
\begin{multline*}
\text{for all $i$, with high probability} \\ 
	|v_i - \frac{n}{2}| = O(\sqrt{t\cdot n\log n}),
\end{multline*}
to saying 
\begin{multline*}
	\text{with high probability, for all $i$} \\ 
	|v_i - \frac{n}{2}| = O(\sqrt{t\cdot n\log n}).
\end{multline*}
This establishes that, with high probability, 
$$v_{max}-v_{min} = O(\sqrt{t\cdot n\log n}).$$
\end{proof}
Using this lemma we can now prove Theorem \ref{thm:stridedAlg}.

\begin{proof}[Proof of Theorem \ref{thm:stridedAlg}]
$ $\\
\paragraph{Span}
There are 2 contributions to the span of our algorithm.
First, there is a contribution of $\frac{n}{t}$ for performing serial partitions on each $P_i$ in parallel because serial partition has a running time that is on the order of the input array, and $|P_i| = \frac{n}{t}$.
Second, there is a contribution from performing a serial partition on the subarray of size $v_{min} - v_{max}$ which is $O(\sqrt{t\cdot n\log n})$ by Lemma \ref{lem:epsSmallFP}.
Thus the total span is 
$$T_\infty = O(n/t+\sqrt{t\cdot n \log n}).$$ \\
We can choose $t$ in order to minimize span.
To minimize span, ignore the log factor and set 
$$n/t = \sqrt{t\cdot n} \implies t=n^{1/3}.$$
This yields the minimum span, because if either term could be decreased, then the function would not be at a minimum, so the terms must be equal.
This yields a span of $O(n^{2/3}).$

\paragraph{Cache Misses}
The number of cache misses in a serial partition is the input size, so the total number of cache misses in this program is $t \cdot \frac{n}{t}$ from the serial partitioning of the $P_i$s in parallel, plus $O(\sqrt{t\cdot n\log n})$ for when the algorithm performs a serial partition on the subarray. So the total number of cache misses is $$n + O(\sqrt{t\cdot n\log n}).$$
To show that the second term in this sum becomes insignificant compared to $n$ as $n$ grows, we take 
$$\lim_{n\to\infty}\frac{\sqrt{t\cdot n \log n}}{n} = \sqrt{\lim_{n\to\infty}\frac{t\cdot \log n}{n}}.$$
We can't choose $t>n$, and making $t = o(\frac{n}{\log n})$ is a reasonable constraint which guarantees that,
$$\lim_{n\to\infty}\frac{t}{\paren*{\frac{n}{\log n}}} = 0.$$
Thus, the algorithm only incurs $n+o(n) = n(1+o(1))$ cache misses.
\end{proof}

\paragraph{Spatial locality}
The Strided Algorithm as described above is good because of the low number of cache misses, but it is not good in that there is no spatial locality in referencing elements of each $P_i$.
A way to increase spatial cache friendliness of the algorithm is to redefine the parts $P_i$s using blocks of adjacent elements instead of individual elements each separated by $t$ indices for each $P_i$.
Let $b$ be the \defn{block size}. \\
To compute $v_i$, the index into $A$ of the first successor in the partitioned $P_i$, we must now compute which block the index $x_i$ into $P_i$ falls in, denoted by $w_i$ and where within the block $x_i$ falls, denoted $m_i$ as follows:
$$w_i = \floor*{ \frac{x_i}{b} },\,\, m_i=x_i \mod b.$$
Once we have $w_i$ and $m_i$ we compute $v_i$ as follows:
$$v_i = t\cdot b\cdot w_i + (i-1)b+m_i.$$
But we don't need that much detail.
More simply this is, 
$$v_i = O(t\cdot b \cdot w_i) = O\paren*{x_i\cdot t}.$$
Where the absorption into the big-O notation is reasonable because $m_i < b,\,\, i<t$ so the other terms get absorbed in the big-O notation, and we can also drop the floor and cancel the $b$. 
Note that this is the same expression we got for the conversion from $x_i$ to $v_i$ as last time (in big-O notation).
So we once again get the same bound on $v_{max}-v_{min}$, which doesn't actually depend on $b$, that is, with high probability
$$v_{max}-v_{min} = O(\sqrt{t\cdot n\log n}).$$

\paragraph{Summary}
It is not fully satisfactory that this algorithm only has these theoretical guarantees for the restrictive case of random inputs.
We use the ideas of this algorithm to create a new algorithm that has have theoretical guarantees on arbitrary inputs.

\section{Key Algorithmic Ideas}
We add randomization to the Strided Algorithm to make a new algorithm that has theoretical guarantees for arbitrary inputs while remaining cache efficient and in-place.
To accomplish this the algorithm forms groups $G_0,G_1,\ldots,G_{g-1}$ which are each a collection of indices to parts $P_j$. 
We will refer to the set that is the union of all parts $P_j$ where $j \in G_y$ as $U_y$. That is,
$$U_y = \bigcup_{j\in G_y}P_j.$$
Then the algorithm partitions each $U_y$ in serial.
Define $\mu_y$ to be the fraction of $U_y$ that is composed of predecessors.
It is important that $|\mu_y - \mu|$ is small for all $y$ with high probability in $n$ so that the algorithm will have a small subproblem to recurse on.
We present 3 versions of the algorithm which are differentiated by their methods of forming groups.

\paragraph{Version 1}
In this version of the algorithm, we form groups $G_0,G_1,\ldots,G_{g-1}$ of indices for certain $P_js$, which are defined in the same way that they were defined in the Strided algorithm, as follows:
\begin{itemize}
	\item Permute the numbers $0,\ldots, t-1$ into $g$ groups of size $O(\polylog n)$.
	\item Sort each group $G_y$ of indices in serial. 
	\item For every group $G_y$ perform a serial partition on $U_y$ the union of all $P_j$ where $j \in G_y$. To accomplish this, the algorithm must find the next or previous element in a group starting from the index of some other element in the group. This is accomplished by storing both the index into whichever part $P_j$ the algorithm is currently on, and the index into the group $G_y$ specifying which part $P_j$ the algorithm is currently on. Then to, for example, find the next element the algorithm increments through the part $P_j$ that it is currently in, until it reaches the end of the part, after which the algorithm increments its counter into $G_y$ and starts at the beginning of the next part $P_j$.
	\item Partition $A[v_{min}], \ldots, A[v_{max}-1]$ where $v_{min}=\min{v_j}, v_{max}=\max{v_j}$, and $v_j$ is defined, as in the Strided algorithm, to be the index of the first successor in $P_j$. As before, this guarantees that all elements of $A$ with indices $k<v_{min}$ are predecessors, and all elements with indices $k\geq v_{max}$ are successors.
\end{itemize}
This algorithm has the desirable property that, because we randomly group $O(\polylog n)$ $P_j$s together, the fraction $\mu_y$ of $U_y$ that is composed of predecessors should cluster closely around the fraction $\mu$ of $A$ that is composed of predecessors for all $y$.
Also, we can make the algorithm have small span by performing some of the steps in parallel.
The permutation step can be done in parallel. 
The algorithm can also perform the serial partition of all collections $U_y$ in parallel.
The unsatisfactory property of this version of the algorithm is the auxiliary memory that is required by the algorithm.
The algorithm stores $g$ groups of size $O(\polylog n)$ where $g$ is chosen to be $O(n / \polylog n)$, so the algorithm uses $O(n / \polylog n)$ extra space and is not in-place.

\paragraph{Version 2}
In this version of the algorithm we again assign the indices of parts $P_0,\ldots, P_{t-1}$ to groups $G_0,\ldots, G_{g-1}$.
However, in this version the groups are not independent of one another.
Because of this dependence, the algorithm only needs $O(\polylog n)$ auxiliary space to store all of the groups.
The algorithm creates an array $X$ of size $O(\polylog n)$ and sets each $X[i]$ to be an integer chosen uniformly at random from $[0,g-1]$.
The value $X[i]$ is used to determine a unique index $j \in [i\cdot g, (i+1)\cdot g-1]$ for each group indicating which part $P_j$ belongs to the group $G_y$ from this set of indices.
The index for the part from this set of indices that belongs to $G_y$ is $$i \cdot g + ((X[i]+y)\bmod g).$$ 
This method of assigning parts to groups assigns all $P_j$s to exactly 1 group.
Having independent random variables for each group is not necessary because we apply a union bound over all groups in proving that $|\mu - \mu_y|$ is small for all $y$ with high probability in $n$, and the union bound still applies for random variables that are not independent.
This version of the algorithm is an improvement over the previous algorithm because it does not use excessive auxiliary memory and it has small span, similarly to that of the previous algorithm. 
An undesirable property of this algorithm however is that it has some unnecessary overhead.

\paragraph{Version 3}
In this version of the algorithm we eliminate the use of parts $P_j$ in the algorithm.
The set $U_y$ is a union of parts $P_j$ which each contain multiple elements.
The motivation for having $P_j$ contain multiple elements was that in the Strided Algorithm the $P_j$s were the only place where we combined elements into a set to partition in serial in the algorithm.
We need to have collections of elements to make each collection have approximately the same fraction of predecessors as the entire array $A$.
Now however, we are combining elements on 2 levels: we group elements into $P_j$s, and we group $P_j$s together into $G_y$s.
We only need one source of grouping elements though, so we can set $|P_j| = 1$ and the collections $U_0,\ldots,U_{g-1}$ will still contain a collection of elements.
Thus the groups can still be expected to have close to the same fraction $\mu_y$ of predecessors in them as the fraction $\mu$ in the entire array $A$.
Setting $|P_j|=1$ makes $P_j=\{A[j]\}$, demonstrating that $P_j$ is no longer needed because it refers to a single element of $A$.
When the cache block size is $b$ instead of $1$ it is convenient to still have parts $P_j$s and say that each is composed of a single cache block.
Thus, we now refer to $P_j$ as a block, and it is defined to contain elements $A[b\cdot j],\ldots, A[b\cdot (j+1) - 1]$.
Changing the role of the $P_j$s is desirable because it adds simplicity to the algorithm and reduces overhead without sacrificing any desirable characteristics of the algorithm.

\section{Analysis of Grouped Partition}
In this section we give a more detailed description of our final algorithm, which we call the \defn{Grouped Partition Algorithm} and prove some key results regarding its performance and number of caches misses. 

\paragraph{Algorithm Overview}
We logically divide the array $A = A[0],A[1],\ldots, A[n-1]$ into blocks $P_j$ each of $b$ adjacent elements, where $b$ is the \defn{block size}.
That is, 
$$P_j = \{A[b\cdot j], A[b\cdot j + 1], \ldots, A[b\cdot (j+1)-1]\}.$$
Note that the $P_j$s are defined in a different way then in the Strided algorithm when $P_j$ contained elements spaced throughout the array.
This is equivalent to setting $|P_j| = 1$ for the $P_j$s in the Strided Algorithm and then making $P_j$ a collection of cache blocks.
Define \defn{$\pred(j)$} to be the number of predecessors in $P_j$.
There are $n/b$ blocks $P_j$.

The algorithm will form $g$ groups of these $P_j$s by the following procedure. 
The algorithm generates a random array $X=X[0],X[1],\ldots,X[s-1]$ where each element in $X$ is an integer chosen randomly at uniform from $[0,g-1]$.
The values in $X$ determine groups $G_0, G_1, \ldots, G_{g-1}$ of $P_j$s.
The first group is 
\begin{multline*}
	G_0 = \{X[0], X[1]+g, X[2]+2\cdot g,\\ 
	\ldots,X[s-1]+ (s-1)\cdot g\}.
\end{multline*}
This means that group $G_0$ is a collection of indices for specific parts $P_j$, indicating that these parts belong to group $G_0$. Similarly, group $G_y$ is defined as 
\begin{multline*}
	G_y = \{(X[0]+y) \bmod g, (X[1]+y) \bmod g + g, \\
	\ldots, (X[s-1]+y) \bmod g + (s-1)\cdot g \}.
\end{multline*}
Intuitively this means that, for group $G_y$, on each chunk of size $g$ of the array, take $X[j]$ and add the group's index $i$ (wrapping around if there is overflow beyond the number of groups) to get to the index of the $P_j$ that belongs to group $G_y$ from this chunk of the array.
Note that we do not need to store the indices of each $P_j$ that belongs to a group because $X$ and the group index is enough information to determine which $P_j$s belong to a group.
We call our algorithm in-place because $s$ (recall $s = |X|$) is made $O(\polylog(n))$.

Define $U_y$ to be the union of all parts that belong to group $G_y$. That is,
$$U_y = \bigcup_{j\in G_y} P_j.$$
Define $\mu_y$ to be the number of predecessors in $U_y$ divided by the number of elements in $U_y$. This is analogous to the definition of $\mu$: the number of predecessors in $A$ divided by $n$.

Once the algorithm has generated $X$, it performs a serial partition on each collection $U_y$ in parallel. 
The algorithm performs the partitions in parallel by splitting the tasks using a recursive method of calling the function repeatedly.
While performing the serial partitions of each $U_y$, the algorithm computes the indices $v_{max}, v_{min}$ where $v_{max}=\max{v_y}, v_{min}=\min v_y$ and $v_y$ is the index in $A$ of the first successor in $U_y$.
This is roughly outlined in Figure \ref{fig:recursiveParallelLoop}.
\begin{figure*}
	\caption{Recursive Spawning Implementation of Parallel For Loop}
	\label{fig:recursiveParallelLoop}
	\textbf{Input:} first group index, number of groups to process\\
	\textbf{Output: $v_{min}, v_{max}$ for $U_y,\ldots, U_{y+n-1}$}
	\begin{algorithmic}[1]
		\Procedure{partitionGroups}{$y, n$}
			\If{$n=1$}
				\State perform serial partition of $U_y$
				\State\Return{$v_y,v_y$}
			\Else
				\State leftvs $\gets$ \Call{partitionGroups}{$y,n/2$}
				\State rightvs $\gets$ \Call{partitionGroups}{$y+n/2,n/2+n\&2$}
				\State\Return{$\min(\text{leftvs.$v_{min}$}, \text{rightvs.$v_{min}$})$, $\max(\text{leftvs.$v_{max}$}, \text{rightvs.$v_{max}$})$}
			\EndIf
		\EndProcedure
	\end{algorithmic}	
\end{figure*}

Then, because for all $y$ 
$$v_{min}\leq v_{y} \leq v_{max},$$
all elements of $A$ with index less than $v_{min}$ are predecessors and all elements of $A$ with index greater or equal to $v_{max}$ are successors.
Thus, by recursing on the subarray $A[v_{min}],\ldots,A[v_{max}-1]$, we complete the partitioning of the array.
We recursively apply the Grouped Partition algorithm to the subproblem.
The base case for the recursion is that when the algorithm can no longer make a substantial number of groups, in which case it partitions the input array in serial.

See Figure \ref{alg:parallelPartition} for a pseudocode implementation of the algorithm.

%%%%%%%% ALEK ALEK ALEK: show recursive spawning rather than parallel for loop?
\begin{figure*}
	\caption{Parallel Partition}
	\label{alg:parallelPartition}
	\begin{algorithmic}[1]
		\If{$g<2$}
			\State serialPartition A
		\Else
			\For{$i \in \{0,1,\ldots,s-1\}$}
				\State $X[i] \gets$ a random integer from $[0,g-1]$ 
			\EndFor
			\State -- We implement this parallel for-loop with a sequence of recursive spawns, and which facilitates computing $v_{min}$, $v_{max}$ without storing $v_y$s
			\ForAll{$ y \in \{0, 1,\ldots,g-1\}$ in parallel}
				\State -- Now we perform a serial partition on $U_y$

				\State -- Initialize ALowIdx to be the index of the first element in $U_y$
				\State $\text{ALowIdx} \gets ((X[0]+y) \bmod g)\cdot b$

				\State -- Initialize AHighIdx to be the index of the last element in $U_y$
				\State $\text{AHighIdx} \gets n - g\cdot b + ((X[s-1]+y) \bmod g)\cdot b + b-1$

				\While{ALowIdx $<$ AHighIdx} 
					\While{A[ALowIdx] $\leq$ pivotValue}
						\State ALowIdx $\gets$ ALowIdx$+1$
						\If{ALowIdx on block boundary}
							\State -- We perform a block increment
							\State $i \gets $ \# of block increments so far (including this one)
							\State -- Increase ALowIdx to start of block $i$ of $G_y$
							\State $\text{ALowIdx} \gets ((X[i] + y)\bmod g) \cdot b + i\cdot b\cdot g$
						\EndIf
					\EndWhile
					\While{A[high] $>$ pivotValue}
						\State AHighIdx $\gets$ AHighIdx$-1$
						\If{AHighIdx on block boundary}
							\State -- We perform a block decrement
							\State $i \gets $ \# of block decrements so far (including this one)
							\State -- Decrease AHighIdx to end of block $s-1-i$ of $G_y$
							\State $\text{AHighIdx} \gets ((X[s-1-i] + y)\bmod g) \cdot b + i\cdot b\cdot g + b - 1$
						\EndIf
					\EndWhile

					\State Swap $A[\text{ALowIdx}]$ and $A[\text{AHighIdx}]$
				\EndWhile
			\EndFor
			\State Recurse on $A[v_{min}],\ldots,A[v_{max}-1]$
		\EndIf
	\end{algorithmic}	
\end{figure*}

Now we prove the following lemma to simultaneously bound $|\mu - \mu_y|$ for all $y$. 
\begin{lemma}
Given $n, \mu$, For our choice of $b, \delta$, if $s=\Theta(\frac{\log n}{\delta^2})$ and $g = \frac{n}{b\cdot s}$ then the Grouped Partition Algorithm forms $g$ such that, with high probability in $n$, for all $y$, $|\mu - \mu_y| < \delta$.
\label{lem:groupedAlgDelta}
\end{lemma}

\begin{proof}[Proof of Lemma \ref{lem:groupedAlgDelta}]
$ $\\
First we use Hoeffding's inequality (i.e. Chernoff Bounds for random variables on $[0,1]$ rather than on $\{0,1\}$) to bound the probability that $|\mu - \mu_y| \geq \delta$. 
We can express $\mu_y$ as the sum of the number of predecessors in all the blocks of the group $G_y$ divided by $|G_y|\cdot|P_j|$.
Note that $G_y$ is composed of $s$ blocks of $b$ elements each, so $|G_y|\cdot|P_j| = s\cdot b$.
Thus we can express $\mu_y$ as \\
\begin{equation}
	\begin{split}
	\mu_y = \frac{1}{|G_y|\cdot|P_j|}\sum_{j \in G_y} \pred(j) \\
	= \frac{1}{b \cdot s}\sum_{i=0}^{s-1}\pred(G_y[i]).
	\end{split}
	\label{eqn:emui}
\end{equation}
We are interested in calculating that the expected value of $\mu_y$, or equivalently calculating the average fraction of predecessors in group $G_y$ over all allowable assignments of blocks to the group.
This is, 
$$\E(\mu_y) = \E\paren*{\frac{1}{s\cdot b} \sum_{i=0}^{s-1}\pred(G_y[i])}.$$
By linearity of expectation this is
$$\E(\mu_y) = \frac{1}{s\cdot b} \sum_{i=0}^{s-1}\E(\pred(G_y[i])).$$
Because $G_y[i]$ is an integer random variable which assumes an index uniformly at random from $[g \cdot i, g \cdot (i + 1) - 1]$ we can expand the expectation to:
$$\E(\mu_y) = \frac{1}{s\cdot b}\sum_{i=0}^{s-1}\sum_{k=0}^{g-1}\frac{1}{g}\pred(k+i\cdot g).$$
	But the argument to $\pred$ assumes all possible values from $[0, s\cdot g -1]$, so we can rewrite the sum as
$$\E(\mu_y) = \frac{1}{b\cdot s \cdot g}\sum_{j=0}^{s\cdot g-1}\pred(j) = \frac{n \cdot \mu}{b \cdot s \cdot g} = \mu.$$ 
We are looking for the probability that 
$$\Big | \mu_y - \mu \Big | \geq \delta.$$
	Note that $\mu_y$ is the average of the $s$ independent random variables $$\frac{1}{b}\pred(G_y[i]) \in [0, 1]$$ by Equation \eqref{eqn:emui}, so we can apply Hoeffding's inequality.
Let $M_y$ be the event that $|\mu_y-\mu| \geq \delta$. 
Hoeffding's inequality gives us that the probability of $\mu_y$ deviating from its expected value, which was calculated to be $\mu$, by more than $\delta$ is
	$$\Pr(M_y) \leq \exp(-2s\delta^2).$$
We have found this bound for each group $G_y$ by themselves, but we need the bound to apply to all groups at once. 
To extend the bound to apply to all groups at once we use the idea of the union bound, which says that if a collection of events each individually do not happen with high probability, then the union of all of these events also does not occur with high probability.
We wish to make $$\Pr\paren*{\bigcup_{y=0}^{g-1} M_y } < \epsilon.$$ 
Using the fact that 
	$$\Pr\paren*{\bigcup_{y=0}^{g-1} M_y } \leq \sum_{y\in [0,g-1]} \Pr(M_y), $$
	we see that by making each $$\Pr(M_y) <\epsilon / g,$$ we can make the probability of any of the events $M_0,\ldots,M_{g-1}$ occurring less than $\epsilon$.
	We established an upper bound for $\Pr(M_y)$ above with Hoeffding's inequality, so to ensure that $\Pr(M_y)$ is less than $\epsilon / g$ it suffices to make the upper bound less than $\epsilon / g$. That is,
	$$\Pr(M_y) < \exp(-2s\delta^2) \leq \frac{\epsilon}{g}.$$
Solving for the necessary size of $s$ we find that  
	$$s \geq \frac{\ln(g / \epsilon)}{2\delta^2}.$$
If we want all events $M_y$ to not occur with high probability in $n$, then we must set $\epsilon = 1/n^c$. 
Thus,
	$$s \geq \frac{\ln(g \cdot n^c)}{2\delta^2}.$$
	%% HEY maybe do the version that is parameterized by epsilon...
Larger values of $s$ hurt our span however, so we will chose $\delta$ to be just big enough to ensure that $M_y$ all happen with high probability in $n$.
Thus we chose 
	$$s = \Theta \paren*{ \frac{\log n}{\delta^2}}$$
	to ensure that with high probability in $n$ ${|\mu - \mu_y| < \delta}$ for all $y$.
\end{proof}

\paragraph{Grouped Partition Subproblem Analysis}
Once the Grouped Partition algorithm has made a single pass over the array there is a subproblem left to solve.
Lemma \ref{lem:groupedAlgDelta} can be used to find out how much smaller we can expect the subarray to be relative to the original array, from which we can derive the span and cache efficiency of the Grouped Partition Algorithm.
There are two options for solving the recursive subproblem, and we calculate the span of the algorithm given each approach.

\begin{proposition}
By first applying the Grouped Partition algorithm to the top level of recursion for the partition problem, and then solving the remaining subproblem problem with an already existing algorithm for parallel partition, the algorithm achieves span
	$$T_\infty(n) = O\paren*{\paren*{\frac{b}{\delta^2} + \log\log n} \log n},$$ and makes $(1+
\delta)\cdot n$ cache misses.
\end{proposition}
\begin{proof}
From previous work we know that it is possible to perform a parallel partition of an array of size $\delta\cdot n$ with span $O(\log(n\delta)\log\log(n\delta)).$ 
With this method our algorithm consists of first sorting each of the $g$ groups in serial, which will take time $O(s\cdot b)$ for each group because serial partition is has linear run time in the input size and $|U_y|=|G_y|\cdot |P_j| = s \cdot b$. 
After this the algorithm will have, with high probability in $n$, a subarray of size $n \cdot \delta$ to partition, which it does in parallel with the other algorithm. 
In computing the span of our algorithm, all of the groups can be partitioned in serial at the same time. 
This makes the span of our algorithm, with this choice of a recursive technique, 
	$$T_\infty(n) = O(s\cdot b + \log(n\delta)\log\log(n\delta)).$$ 
	By Lemma \ref{lem:groupedAlgDelta} we choose $s = \Theta(\frac{\log n}{\delta^2})$, so the span becomes 
	$$T_\infty(n)=  O\paren*{b \cdot \frac{\log n}{\delta ^2} + \log(n\delta)\log\log(n\delta)}.$$
Note that $\log(n\delta) = \log(n) - \log(\frac{1}{\delta}) < \log n$ so we can simplify the span to
	$$T_\infty(n) = O\paren*{\paren*{\frac{b}{\delta^2} + \log\log n} \log n}.$$

	Now we compute the number of cache misses incurred. 
	On the top level, the algorithm incurs $n$ cache misses for the serial partitions of all the groups.
	%%%%%% CHECK THISSSSSSSSSSS
	When recursing on an array of size $\delta \cdot n$ the algorithm incurs $\delta \cdot n$ cache misses. 
	Thus, in total the algorithm incurs $(\delta + 1) \cdot n$ cache misses.
\end{proof}

\paragraph{Recursing With the Same Algorithm}
We could instead recurse with the Grouped Partition algorithm.
Note that with Francis and Pannan's algorithm which only performs well on random inputs we could not do this because their algorithm works poorly on the subarray generated, but our algorithm's performance is guaranteed for arbitrary inputs so we can recurse with our algorithm.
There are two ways we can do this. 
We could applying special treatment to the top level of recursion because it is by far the biggest problem, or we could treat all levels the same.

\begin{proposition}
If we use the same allowable deviation $\delta$ for $|\mu_y - \mu|$ at all levels of recursion then the algorithm has span $$T_{\infty}(n) = \Theta(b\log^2 n),$$ and incurs approximately $2n$ cache misses.
\end{proposition}

\begin{proof}
If we choose some constant $\delta$, then by Lemma \ref{lem:groupedAlgDelta} choosing $s = \Theta(\log n)$ makes ${|\mu - \mu_y| < \delta}$ for all $y$ with high probability in $n$. 
Then,
$$\mu - \delta < \mu_y < \mu + \delta$$
for all $y$, so 
	$$\max{\mu_y} - \min(\mu_y) < 2\delta. $$
%% The evasion of a hard inequality with the theta stuff is probably bad, especially given that the real inequality isn't too nasty and is pretty informative	
%% here is a sketch of the math for the real stuff
%% $$|\mu - \mu_y|<\delta$$
%% index of the group P_j that v_y lies in is 
%% G_y[\mu_y s] = g \mu_y s + ((X[\mu_y s] + y)%g)
%% g\mu_y s \leq G_y[\mu_y s] < g\mu_y s + g
%% v_y is then within one of these blocks
%% ie   g\mu_y s b \leq   v_y < (g \mu_y s +g)b + b
%% note n = gsb
%% so this is    n \mu_y \leq v_y < n \mu_y + (g+1)b
%% \mu_y n \leq v_y < \mu_y n + (g+1)b
%% $$v_{max} < n(\mu +\delta) + (g+1)b$$
%% v_{min} > n(\mu - \delta)
%% v_{max} - v_{min} < 2n\delta +(g+1)b
We can compute $v_y$ from $\mu_y$, because $\lfloor \mu_y\cdot s \rfloor$ gives the number of parts $P_j$ where $j \in G_y$ that are completely filled with predecessors once the collection $U_y$ is partitioned. 
There are $s$ portions of size $g\cdot b$ in the array, so to find the index of the start of the portion of index $\lfloor \mu_y\cdot s \rfloor$, we compute
	$$v_y = \Theta((\mu_y \cdot s)\cdot (g\cdot b)).$$
	But this is just 
	$$v_y = \Theta(\mu_y \cdot n).$$
	Thus, we can compute the difference ${v_{max} - v_{min}}$ as 
	$$v_{max} - v_{min} = \Theta(n \cdot (\max{\mu_y} - \min{\mu_y}))$$
which is
	$$v_{max} - v_{min} = \Theta\paren*{\delta\cdot n}$$
	Thus for an appropriate constant choice of $\delta$ the algorithm could be made to reduce the problem size by half at each step. We can compute the span of the algorithm given this choice of $\delta$
$$T_{\infty}(n) = \Theta(b\cdot s)+T_{\infty}(n/2).$$
	Note that the $\Theta(b\cdot s)$ term incurred from the serial partitioning of the groups in parallel can be re-expressed as $\Theta(b \log n)$ because we chose $s=\Theta(\log n)$.
Thus we can recursively express the span as 
$$T_{\infty}(n) = \Theta(b\cdot \log n)+T_{\infty}(n/2).$$
	The depth of the recursion is $\Theta(\log n)$ because the input size is reduced by half at each level. Thus, by adding up the $\log n$ terms of the form $$b \cdot \log (n/2^i) = b\cdot (\log n - i), $$
we get the span of the algorithm, which is 
	$$T_{\infty}(n) = \Theta(b\log^2 n).$$

	The number of cache misses is approximately
	$$n + n/2^1 + n/2^2 + \cdots = 2n .$$
\end{proof}

\begin{proposition}
	If we take $\delta = O(1 / \sqrt{\log n})$ on the top level of recursion and then use a constant $\delta$ on lower levels of recursion, then the algorithm has span $\Theta(b\log^2 n)$.
\end{proposition}
\begin{proof}
We could potentially achieve better performance by using a smaller $\delta$ initially, and then using a larger $\delta$ at lower levels of recursion.
This means that the algorithm will reduce the size of the problem significantly on the top level of recursion and then use a $\delta$ that does not hurt its span on lower levels where it is not critical that the size gets reduced a lot.
We can compute the span of the algorithm when run on the initial input $n_0$,
$$T_\infty(n_0) = b\cdot \Theta(\frac{\log n_0}{\delta ^2})+T_\infty(\delta n_0)$$
and for all $n<n_0$, where we use $\delta_{fake} = \frac{1}{2}$
$$T_\infty(n) =  b\cdot \Theta(\log n_0)+T_\infty(\delta_{fake} n).$$
From the previous part we have shown that once we are using $\delta_{fake}$ we get span $\Theta(b\log^2 n)$, so we can have $\delta = \frac{1}{\sqrt{\log n}}$ to get span $\Theta(b\log^2 n)$, but this is much better than before. 
The improvement can be seen in the cache behavior analysis.
The algorithm incurs $n$ cache misses on the top layer, and then at lower layers the algorithm incurs less than
	$$\frac{1}{\sqrt{\log n}} + \frac{1}{2\sqrt{\log n}} + \cdots = \frac{2}{\sqrt{\log n}}.$$
	Thus, the total number of cache misses is $$(1+o(1))n$$ because $\frac{2}{\log n} \in o(1)$.

\end{proof}
	
	
\section{Systems Performance}

\paragraph{Handling Non-divisible Array Size}
In the analysis above we have assumed that $n$ is divisible by our parameters $g, b, s$. 
However, this simplifying assumption is incredibly restrictive and unnecessary.
In our algorithm $b$ and $s$ parameters, and then $g$ is determined as $$g = \floor* { \frac{n}{b\cdot s} }.$$
If $n \bmod (b\cdot s) \not\equiv 0$, then $g\cdot b \cdot s < n$.
To deal with this problem, we first apply our partial partition algorithm to ${A[0],\ldots,A[g\cdot s\cdot b -1]}$.
After doing this there is a subarray ${A[v_{min}],\ldots, A[v_{max}-1]}$ left to partition, and there is also the part of the array ${A[g\cdot s\cdot b],\ldots, A[n-1]}$ that we have ignored.
The algorithm deals with the tail of the array by either swapping the tail with a portion of the array of equal size which is the earliest successors in the array, or, in the rare case that there are more elements in the tail than successors--this may realistically happen on a very low layer of recursion when solving the partition problem with repeated applications of our algorithm, or if the pivot value is chosen adversarially such that there are very few successors--then the entire section of the array which is composed of successors is swapped with the last portion of the tail which is of the same length as the length of the section of the array composed of successors and appropriately incrementing $v_{max}$. 
This step adds very little to the size of the recursive subproblem because $$n - b\cdot s \floor*{\frac{n}{b\cdot s}} < b\cdot s.$$
This does not affect the algorithm's work, span, or cache-efficiency.

\paragraph{Optimizations}
The Grouped Partition algorithm must perform a serial partition on each group $G_y$.
The serial partition is complicated by the fact that the groups do not contain a single contiguous portion of the array, but rather they contain $s$ blocks $P_j$ of size $b$ each.
In the serial partition the algorithm must repeatedly find the next or previous element in the group.
Although the algorithm could utilize modular arithmetic and floor division of a counter to find the index into the array $A$, division and modulo are very expensive operations to perform, so we use the better approach of checking proceeding within a block while checking if we have hit the boundary of the block, and then at that point jumping to the next block.
It is possible with this method to only use addition and subtraction and bit shifting by $\log b$, which is numerically equivalent to multiplication by $b$ but faster and possible because we choose $b$ to be a power of $2$.
This makes the determination of next and previous elements in the serial partition step much faster.
Our algorithm spends most of its time on the serial partition step at the top level of recursion.
This makes it so that the algorithm gets high parallelism because by increasing the number of groups that the algorithm can process in parallel the amount of time that the algorithm must spend on the time consuming step of serial partitioning is reduced.
The overhead introduced in the serial partition step of the algorithm slows it down initially, but is overcome when the number of processors increases.

\paragraph{Performance}
This new algorithm is very cache efficient, and operates very close to the memory bandwidth bound. Our implementation of the algorithm is available on github: https://github.com/awestover/Parallel-Partition.  

\end{document}
