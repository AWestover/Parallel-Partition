
% Alek Westover writeup draft 

\documentclass[twocolumn, twoside, 12pt]{article}
\author{\vspace{-2ex}Alek Westover\vspace{-2ex}}
\title{\vspace{-8ex}Cache Efficient Parallel Partition\vspace{-2ex}}
\date{\vspace{-2ex}June 26, 2019\vspace{-2ex}}

\renewcommand{\thesubsection}{\thesection.\alph{subsection}}
\usepackage{graphicx}
\usepackage{listings}
\usepackage{amsfonts}

\usepackage{amsmath}
% defines a new function
\def\polylog{\operatorname{polylog}}
\def\pred{\operatorname{pred}}
\def\E{\operatorname{\mathbb{E}}}

\usepackage{mathtools}
\DeclarePairedDelimiter{\floor}{\lfloor}{\rfloor}
% \floor*{\frac{x}{y}} actually does a nice looking floor!
\DeclarePairedDelimiter{\paren}{(}{)}
% paren*{ } gives the correct size
\newcommand{\defn}[1]       {{\textit{\textbf{\boldmath #1}}}}

\renewcommand{\paragraph}[1]{\vspace{0.09in}\noindent{\bf \boldmath #1.}} 

\usepackage{amsthm}
\usepackage{amssymb}
\usepackage{algorithm, algpseudocode}

%%%% Theorem stuff
\newtheorem{theorem}{Theorem}
\newtheorem{lemma}{Lemma}
\newtheorem{proposition}{Proposition}

\begin{document}
\maketitle
\begin{abstract}
	This is a modified version of the "Algorithm Overview" paragraph in the "Analysis of Grouped Partition" section of the other writeup that I am doing. 
\end{abstract}

\paragraph{Algorithm Overview}
We now describe the \defn{Grouped Partition} algorithm.
We logically divide the array $A = A[0],A[1],\ldots, A[n-1]$ into blocks $P_j$ each of $b$ adjacent elements, where $b$ is the \defn{block size}.
Note that the $P_j$s are defined in a different way then in the Strided algorithm when $P_j$ contained elements spaced throughout the array.
This is equivalent to setting $|P_j| = 1$ for the $P_j$s in the Strided Algorithm and then making $P_j$ a collection of cache blocks.
Define \defn{$\pred(j)$} to be the number of predecessors in $P_j$.
There are $n/b$ of these blocks $P_j$.

The algorithm generates a random array $X=X[0],X[1],\ldots,X[s-1]$ where each element in $X$ contains an integer chosen randomly at uniform from $[0,g-1]$, where $g$ is the number of groups.
The values in $X$ determine $g$ groups $G_0, G_1, \ldots, G_{g-1}$ of $P_j$s.
The first group is 
\begin{multline*}
	G_0 = \{X[0], X[1]+g, X[2]+2\cdot g,\\ 
	\ldots,X[s-1]+ (s-1)\cdot g\}.
\end{multline*}
This means that group $G_0$ is a collection of indices for specific parts $P_j$, indicating that these parts belong to group $G_0$. Similarly, group $G_y$ is defined as 
\begin{multline*}
	G_y = \{(X[0]+i) \bmod g, (X[1]+i) \bmod g + g, \\
	\ldots, (X[s-1]+i) \bmod g + (s-1)\cdot g \}.
\end{multline*}
Intuitively this means that, for group $G_y$, on each chunk of size $g$ of the array, take $X[j]$ and add the group's index $i$ (wrapping around if there is overflow beyond the number of groups) to get to the index of the $P_j$ that belongs to group $G_y$ from this chunk of the array.
Note that we do not need to store the indices of each $P_j$ that belongs to each group because between $X$ and the group index the $P_j$s that belong to the group $G_y$ are uniquely determined.
We call our algorithm in-place because the size $s$ of X is made $\Theta(\frac{\log n}{\delta^2})$, for $\delta \in (0, 1)$ of our choice, which is minuscule compared to the array of size $n$ that is an input to the partition problem.

Define $U_y$ to be the union of all parts that belong to group $G_y$, that is
$$U_y = \bigcup_{j\in G_y} P_j.$$
Define $\mu_y$ to be the number of predecessors in $U_y$ divided by the number of elements in $U_y$. This is analogous to the definition of $\mu$: the number of predecessors in $A$ divided by $n$.

Once the algorithm has generated $X$, it performs a serial partition on each collection $U_y$ in parallel. 
After partitioning the collection $U_y$ the Grouped Partition algorithm stores the index $v_y$, which is the index in $A$ of the first successor in $G_y$, in array of size $s$.
Recall that $s$ is small, and that the algorithm already created an array $X$ of size $s$, so this does not change the asymptotics for memory consumption of the algorithm.
After the Grouped Partition algorithm finishes performing in parallel a serial partition of each group, the algorithm computes in serial $v_{min} =\min v_y$, and $v_{max} = \max{v_y}$ from the stored values $v_y$ generated by each serial partition.
Note that we cannot just have a variable that all threads can write to that stores the current lowest $v_y$ and current highest $v_y$ and then overwrite this in each thread when it finishes its group as we would try in a serial algorithm because this would cause data races in the parallel algorithm.
Thus, because for all $y$ 
$$v_{min}\leq v_{y} \leq v_{max},$$
we can determine that all elements of $A$ with index less than $v_{min}$ are predecessors and elements of the array with index greater or equal to than $v_{max}$ are successors.
Thus, by recursing on the subarray $A[v_{min}],\ldots,A[v_{max}-1]$, we complete the partitioning of the array.
We recursively apply the Grouped Partition algorithm to the subproblem.
The base case for the recursion is that when the algorithm can no longer make a substantial number of groups, in which case it partitions its input array in serial. 

\end{document}
